\newpage
\section{Wstęp}

Współczesne konstrukcje, materiały i koncepcje budowlane stają się coraz bardziej złożone, wymagając od inżynierów niezwykłej precyzji i uwagi na każdy detal.
Los zarówno budowli, jak i ich użytkowników, zależy od najmniejszych decyzji projektowych.
W dobie komputerów, ręczne obliczenia i analizy ustępują miejsca zaawansowanym narzędziom programistycznym, które zapewniają nieosiągalną wcześniej dokładność i efektywność.

Rozwój programowania w drugiej połowie XX wieku zrewolucjonizował wiele dziedzin, w tym inżynierię lądową.
Programy takie jak AutoCAD, Robot Structural Analysis czy Revit umożliwiają projektowanie różnorodnych konstrukcji z niespotykaną dotąd szybkością i precyzją.
Niemniej jednak, każdy projekt wymaga wnikliwej analizy, a wprowadzane zmiany mogą pociągać za sobą konieczność żmudnych, ręcznych przeliczeń.
W tym kontekście, automatyzacja procesu analizy staje się kluczowa dla optymalizacji czasu pracy inżyniera oraz minimalizacji ryzyka błędów.

Niniejsza praca inżynierska koncentruje się na analizie wrażliwości konstrukcji ramowych, uwzględniając nieodłączny element rzeczywistości inżynierskiej – losowość parametrów geometrycznych i materiałowych.
Celem jest wykorzystanie dostępnych narzędzi programistycznych do przeprowadzenia kompleksowej analizy, która pozwoli lepiej zrozumieć wpływ zmienności tych parametrów na zachowanie konstrukcji.

W pierwszym rozdziale przedstawiono teoretyczne podstawy analizy wrażliwości oraz scharakteryzowano wykorzystane narzędzia programistyczne.
Drugi rozdział zawiera szczegółowy opis stworzonego skryptu do analizy konstrukcji.
W trzecim rozdziale zaprezentowano modele analizowanych konstrukcji oraz wyniki przeprowadzonych symulacji.
Pracę zamyka podsumowanie oraz bibliografia.

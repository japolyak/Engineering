\newpage
\addcontentsline{toc}{section}{Wstęp}
\section*{Wstęp}

Idąc z duchem czasu konstrukcje, materiały i pomysły budowlane stają się coraz bardziej skomplikowane, co z kolei wymaga
od projektantów większego poświęcenia się szczegółom wykonywanego zadania. Od najmniejszych drobiazgów zależy los zarówno konstrukcji, jak i ludzi,
którzy będą ję użytkować. W dzisiejszych czasach mało kto może mierzyć się siłą z komputerami w kwestiach precyzji
i eliminacji błędów.

\par Druga połowa XX wieku wiąże się z początkiem rozwoju mało popularnej, ale obiecującej branży — programowanie.
Owa dziedzina pomogła ludzkości osiągnąć niebywałe sukcesy w medycynie, kosmosie, nauce i między innymi budownictwie.
Powszechnie używane w Polsce programy inżynierskie — AutoCAD, Robot Structural Analysis, SOLDIS PROJEKTANT, GEO5, Revit i inne,
pozwalają użytkownikom projektować wszystkie możliwe konstrukcje szybko i precyzyjnie. Jednakże każdy proces projektowy wiąże się z
analizą otrzymanych wyników, które najprawdopodobniej będą miały wpływ na początkowe założenia. Ręczne przeliczenie konstrukcji z
uwzględnieniem nowych rozwiązań konstruktorskich albo materiałów jest nie tylko kosztowne, ale również czasochłonne, co może skutkować
pojawieniu się błędów w obliczeniach. Wcześniej wspomniane programy pozwalają w szybki i prosty sposób wykonywać ten sam
proces obliczeniowy, zmieniając jedynie podawane parametry.

Rozważana jest rama przestrzenna składająca się z trzech elementów belkowych.
Wszystkie elementy wykonane są z jednorodnego materiału, o jednakowych właściwości fizyczne oraz geometryczne, takie jak moduł sprężystości
\textbf{E} moment bezwładności przekroju \textbf{I}  oraz powierzchnia przekroju poprzecznego \textbf{A}, są jednorodne.

Rama jest wsparta na dwóch podporach: po lewej stronie znajduje się podpora przegubowo nieprzesuwna, a po prawej wspornik.
Konstrukcja doznaje działania sił skupionych $P_1$ (pozioma w prawo) oraz $P_2$ (pionowa w dół) w lewym górnym narożniku (węzeł 2).

Głównym celem analizy jest ocena wpływu tych obciążeń na przemieszczenie środka belki w kierunku poziomym (w osi X).
W badaniu analizowano, jak losowe zmiany parametrów wpływają na przemieszczenia oraz jak wyniki konwergują przy różnych liczbach prób.

\begin{table}[H]
    \centering
    \begin{tabular}{|c|c|c|c|c|c|c|c|c|}
        \hline
        $L$ & $H$ & $b$ & $d$ & $A$ & $I_z$ & $E$ & $P_1$ & $P_2$ \\
        \hline
        \multicolumn{4}{|c|}{$m$} & $m^2$ & $m^4$ & $GPa$ & \multicolumn{2}{|c|}{$kN$} \\
        \hline
        3 & 4 & 0.3 & 0.6 & 0.18 & $145.8*10^-6$ & $30*10^6$ & 25 & -500 \\
        \hline
    \end{tabular}
    \caption{Parametry ramy}
    \label{tab:pars-rama}
\end{table}

\resultstable
{1 & 2 & 3 & 4 & 5 & 6}
{1 & 2 & 3 & 4 & 5 & 6}
{1 & 2 & 3 & 4 & 5 & 6}
{Wyniki symulacji dla ramy}
{rama-results}

\opsfigure{Rama_model}{Model konstrukji – Rama}
\opsfigure{Rama_deformation}{Deformacja konstrukji – Rama}

\opsfigure{Rama_10000_sp}{Ugięcie środka belki – Rama 10000}
\opsfigure{Rama_10000_zsp}{Zbieżność wartości średniej przemieszczenia - Rama 10000}
\opsfigure{Rama_10000_zos}{Zbieżność odchylenia standardowego – Rama 10000}

\opsfigure{Rama_100000_sp}{Ugięcie środka belki – Rama 100000}
\opsfigure{Rama_100000_zsp}{Zbieżność wartości średniej przemieszczenia - Rama 100000}
\opsfigure{Rama_100000_zos}{Zbieżność odchylenia standardowego – Rama 100000}

\opsfigure{Rama_1000000_sp}{Ugięcie środka belki – Rama 1000000}
\opsfigure{Rama_1000000_zsp}{Zbieżność wartości średniej przemieszczenia - Rama 1000000}
\opsfigure{Rama_1000000_zos}{Zbieżność odchylenia standardowego – Rama 1000000}

Rozważana jest rama przestrzenna składająca się z trzech elementów belkowych.
Wszystkie elementy wykonane są z jednorodnego materiału, o jednakowych właściwości fizyczne oraz geometryczne(moduł sprężystości
\textbf{E}, moment bezwładności przekroju \textbf{I},powierzchnia przekroju poprzecznego \textbf{A}).

Rama jest podparta z lewej strony przegubowo nieprzesuwnie, a z prawej ma wspornik.
Na lewy górny narożnik (węzeł 2) działają dwie siły skupione: $P_1$ (pozioma, skierowana w prawo) i $P_2$ (pionowa, skierowana w dół).
Rysunek \ref{cadmodel:Rama-mod} przedstawia model ramy przestrzennej, a w Tabeli \ref{tab:parametry-rama} zebrane są deterministyczne oraz losowane parametry konstrukcji.
Implementacja obliczeń jest zawarta w załączniku \ref{appendix:Rama}.

Głównym celem analizy jest ocena wpływu tych obciążeń na przemieszczenie narożnika w kierunku poziomym (w osi X).

\cadmodel
{Rama-mod}
{Model ramy przestrzennej}
{0.35}

\begin{table}[H]
    \centering
    \begin{tabular}{|c|c|c|c|c|c|c|c|c|c|}
        \hline
        Rodzaj parametru & L & H & b & d & A & {$\mathrm{I}_\mathrm{z}$} & E & $\mathrm{P}_\mathrm{1}$ & $\mathrm{P}_\mathrm{2}$ \\
        \hline
        – & \multicolumn{4}{|c|}{\textnormal{m}} & [$\mathrm{m}^\mathrm{2}$] & [$\mathrm{m}^\mathrm{4}$] & [GPa] & \multicolumn{2}{|c|}{[kN]} \\
        \hline
        Deterministyczny & 3 & 4 & 0.3 & 0.6 & 0.18 & – & – & – & – \\
        \hline
        Średni & – & – & – & – & – & 145.8 \cdot $\mathrm{10}^{\mathrm{-6}}$& 30 & 25 & –500 \\
        \hline
        Odchylenie standardowe & – & – & – & – & – & 14.6 \cdot $\mathrm{10}^{\mathrm{-6}}$ & 30 & 2.5 & 50 \\
        \hline
    \end{tabular}
    \caption{Parametry ramy}
    \label{tab:parametry-rama}
\end{table}

\constructionresults
{\rama}
{Rama}
{Przemieszczenie narożnika}
{
\resultstable
{1.35 & 0.27 & 1.07 & 0.17 & 31.4 & 0.17}
{1.37 & 0.23 & 1.07 & 0.17 & 30.2 & 0.16}
{1.4 & 0.24 & 1.07 & 0.17 & 30 & 0.16}
{Wyniki przemieszczeń symulacji dla ramy}
{Rama-results}
{[mm] & [mm] & [mm] & [mm] & [nm] & [\%]}
}
{
Rysunek \ref{fig:Rama_model} przedstawia model obliczeniowy konstrukcji ramowej wygenerowany w środowisku symulacyjnym.
Na Rysunku \ref{fig:Rama_deformation} ukazana jest deformacja ramy przestrzennej pod wpływem siły skupionej, przyłożonej w lewym górnym narożniku.
W Tabeli \ref{tab:Rama-results} zebrano wartości przemieszczeń dla każdej z 3 serii obliczeń.
Warto zauważyć, że niezależnie od liczby prób symulacji, uzyskane wyniki przemieszczeń są zbliżone i zgodne z oczekiwaniami.
}
{
Rysunki \ref{fig:Rama_10000_sp}, \ref{fig:Rama_10000_zsp} oraz \ref{fig:Rama_10000_zos} przedstawiają wyniki przemieszczenia narożnika dla 10.000 symulacji.
Średnie przemieszczenie wynosi 1.07 mm, a różnica między maksymalną wartością (1.35 mm) a minimalną (0.27 mm) wynosi 1.08 mm.
Na Rysunkach \ref{fig:Rama_10000_zsp} – \ref{fig:Rama_10000_zos} można zaobserwować, że stabilizacja wyników zaczyna się po około 6.000 symulacji.
}
{
Rysunki \ref{fig:Rama_100000_sp}, \ref{fig:Rama_100000_zsp} oraz \ref{fig:Rama_100000_zos} przedstawiają wyniki dla 100.000 symulacji.
Średnie przemieszczenie wynosi 1.07 mm, a różnica między maksymalną wartością (1.37 mm) a minimalną (0.23 mm) wynosi 1.14 mm.
Na Rysunkach \ref{fig:Rama_100000_zsp} – \ref{fig:Rama_100000_zos} stabilizacja wyników jest widoczna już po około 60,000 symulacji, co świadczy o stopniowym zmniejszaniu się zmienności wyników wraz ze wzrostem liczby prób.
}
{
Dla 1.000.000 symulacji, wyniki przedstawione na  Rysunkach \ref{fig:Rama_1000000_sp}, \ref{fig:Rama_1000000_zsp} oraz \ref{fig:Rama_1000000_zos} pokazują, że średnia wartość przemieszczenia nadal wynosi 1.07 mm, przy czym różnica między wartością maksymalną (1.4 mm) a minimalną (0.24 mm) wynosi 1.16 mm.
Całkowitą stabilizację wyników można zaobserwować po około 100.000 symulacji na Rysunkach \ref{fig:Rama_1000000_zsp} – \ref{fig:Rama_1000000_zos}, co potwierdza, że dalsze zwiększanie liczby prób nie wpływa istotnie na wynik końcowy.
}

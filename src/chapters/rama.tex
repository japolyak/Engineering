Rozważana jest rama przestrzenna składająca się z trzech elementów belkowych.
Wszystkie elementy wykonane są z jednorodnego materiału, o jednakowych właściwości fizyczne oraz geometryczne(moduł sprężystości
\textbf{E}, moment bezwładności przekroju \textbf{I},powierzchnia przekroju poprzecznego \textbf{A}).

Rama jest podparta z lewej strony przegubowo nieprzesuwnie, a z prawej ma wspornik.
Na lewy górny narożnik (węzeł 2) działają dwie siły skupione: $P_1$ (pozioma, skierowana w prawo) i $P_2$ (pionowa, skierowana w dół).

Głównym celem analizy jest ocena wpływu tych obciążeń na przemieszczenie narożnika w kierunku poziomym (w osi X).

\cadmodel
{Rama-mod}
{Model ramy przestrzennej}

\begin{table}[H]
    \centering
    \begin{tabular}{|c|c|c|c|c|c|c|c|c|}
        \hline
        $L$ & $H$ & $b$ & $d$ & $A$ & $I_z$ & $E$ & $P_1$ & $P_2$ \\
        \hline
        \multicolumn{4}{|c|}{$m$} & $m^2$ & $m^4$ & $GPa$ & \multicolumn{2}{|c|}{$kN$} \\
        \hline
        3 & 4 & 0.3 & 0.6 & 0.18 & $145.8*10^-6$ & $30*10^6$ & 25 & -500 \\
        \hline
    \end{tabular}
    \caption{Parametry ramy}
    \label{tab:pars-rama}
\end{table}

\constructionresults
{\belka}
{Belka}
{Ugięcie środka belki}
{
\resultstable
{1.33 & 0.27 & 1.07 & 0.17 & 2.98 & 0.16}
{1.36 & 0.26 & 1.07 & 0.17 & 3 & 0.16}
{1.39 & 0.24 & 1.07 & 0.17 & 3 & 0.16}
{Wyniki symulacji dla ramy}
{rama-results}
{$10^{-3}$ & $10^{-3}$ & $10^{-3}$ & $10^{-3}$ & $10^{-8}$ & $uCov$}
}

\newpage
\section{Wprowadzenie teoretyczne}

\subsection{Analiza wrażliwości}

Analiza wrażliwości jest kluczowym narzędziem w inżynierii konstrukcji, pozwalającym zrozumieć, jak zmiany parametrów wejściowych modelu wpływają na wynik analizowanego problemu.
W praktyce oznacza to, że można ocenić, które parametry - takie jak właściwości materiałowe, wymiary elementów czy obciążenia - mają największy wpływ na zachowanie się konstrukcji pod obciążeniem.
Dzięki temu jest możliwa zidentyfikacja potencjalnych słabości konstrukcji, zoptymalizować projekt pod kątem bezpieczeństwa i ekonomii, a także lepiej zarządzać ryzykiem związanym z niepewnościami w danych wejściowych.

Głównym celem analizy wrażliwości jest ilościowe określenie wpływu zmienności parametrów wejściowych na wybrane kryteria oceny konstrukcji, takie jak przemieszczenia, naprężenia czy ryzyko awarii.
W ten sposób, inżynierzy mogą podejmować świadome decyzje projektowe, uwzględniając nie tylko wartości nominalne parametrów, ale także ich potencjalne odchylenia.

Istnieje wiele metod stosowanych w analizie wrażliwości, różniących się podejściem i złożonością obliczeniową.
Do najpopularniejszych należą:

\begin{itemize}
    \item \textbf{Metoda różnic skończonych (Finite Difference Method)}: Polega na wprowadzeniu małych zmian w wartościach parametrów wejściowych i obserwacji, jak wpływają one na wynik. Jest to prosta metoda, ale może być czasochłonna przy dużej liczbie parametrów.
    \item \textbf{Metoda Monte Carlo}: Stosowana w analizach probabilistycznych, polega na losowym próbkowaniu wartości parametrów wejściowych i analizie wyników. Umożliwia ocenę rozkładu wyników oraz identyfikację najbardziej wpływowych parametrów.
    \item \textbf{Metoda współczynnika czułości (Sensitivity Coefficient Method)}: Oblicza względny wpływ zmiany parametru wejściowego na wynik. Jest to bardziej formalne podejście, które pozwala na dokładniejsze zrozumienie relacji pomiędzy parametrami a wynikami.
    \item \textbf{Metoda Sobola}: Zaawansowana technika, która rozkłada całkowitą wariancję wyników na poszczególne składowe, związane z różnymi parametrami wejściowymi. Pozwala to na ocenę zarówno efektów pojedynczych parametrów, jak i ich interakcji.
\end{itemize}

Analiza wrażliwości znajduje szerokie zastosowanie w inżynierii konstrukcji, szczególnie w kontekście:

\begin{itemize}
    \item Projektowania mostów i budynków pozwalając ocenić, jak zmienność właściwości materiałowych oraz różnorodność obciążeń wpływają na bezpieczeństwo i trwałość konstrukcji, co umożliwia podejmowanie świadomych decyzji projektowych.
    \item Analizy sejsmicznej, gdzie zrozumienie wrażliwości konstrukcji na parametry takie jak masa, sztywność, czy tłumienie jest kluczowe dla oceny ryzyka w przypadku trzęsienia ziemi.
    \item Optymalizacji projektów, umożliwiając identyfikację najbardziej krytycznych aspektów projektu, które mogą być zoptymalizowane w celu zwiększenia efektywności lub zmniejszenia kosztów.
\end{itemize}

\subsection{Losowość geometrii i właściwości materiałowych}

Losowość w kontekście inżynierii odnosi się do nieuniknionych odchyłek od nominalnych wartości parametrów projektowych, które mogą wynikać z różnych źródeł, takich jak:

\begin{itemize}
    \item Wahania w procesie produkcji: Właściwości materiałowe, takie jak moduł sprężystości czy wytrzymałość na rozciąganie, mogą różnić się w zależności od partii materiału lub zmian w procesie produkcji.
    \item Zmiany w geometrii: Mogą one wynikać z tolerancji produkcyjnych, takich jak odchylenia w grubości, szerokości czy długości elementów konstrukcyjnych.
    \item Warunki eksploatacyjne: Czynniki zewnętrzne, takie jak korozja, zużycie, zmiany temperatury, mogą wpływać na geometrię i właściwości materiałowe konstrukcji w czasie.
\end{itemize}

\subsubsection*{Losowość Geometrii}

Geometria konstrukcji może podlegać losowym odchyleniom z różnych powodów:

\begin{itemize}
    \item Tolerancje produkcyjne: Rzeczywiste wymiary elementów konstrukcyjnych mogą różnić się od wartości projektowych w granicach określonych przez normy produkcyjne.
    \item Błędy montażowe: Odchylenia mogą wystąpić również na etapie montażu, gdzie rzeczywiste położenie i orientacja elementów mogą odbiegać od założeń projektowych.
    \item Degradacja i zużycie: Zmiany w geometrii mogą również wynikać z długotrwałej eksploatacji, gdzie elementy konstrukcyjne ulegają deformacjom, pęknięciom lub innym formom degradacji.
\end{itemize}

W kontekście analizy konstrukcji, losowość geometrii może wpływać na:

\begin{itemize}
\item Rozkład naprężeń i sił wewnętrznych: Odchylenia w geometrii mogą powodować nierównomierny rozkład naprężeń, co może prowadzić do lokalnych koncentracji naprężeń.
\item Stabilność konstrukcji: Nawet niewielkie odchylenia od projektowanej geometrii mogą znacząco wpłynąć na stabilność konstrukcji, zwłaszcza w przypadku elementów smukłych lub podatnych na wyboczenie.
\end{itemize}

\subsubsection*{Losowość właściwości materiałowych}

Właściwości materiałowe konstrukcji, takie jak moduł sprężystości \textbf{E}, moment bezwładności \textbf{I}, czy wytrzymałość na rozciąganie \sigma mogą podlegać losowym odchyleniom z kilku powodów:

\begin{itemize}
    \item Niejednorodność materiałów: Materiały inżynierskie, takie jak stal, beton czy drewno, mogą wykazywać naturalną niejednorodność, wynikającą z procesu produkcji lub ze źródeł surowców.
    \item Zmiany środowiskowe: Właściwości materiałowe mogą zmieniać się w zależności od warunków środowiskowych, takich jak temperatura, wilgotność czy ekspozycja na promieniowanie UV.
    \item Proces starzenia: Właściwości materiałowe mogą ulegać degradacji w czasie, co jest szczególnie istotne w kontekście długotrwałej eksploatacji konstrukcji.
\end{itemize}

\subsubsection*{Wpływ Losowości na analizę konstrukcji}

Uwzględnienie losowości w analizie konstrukcji pozwala na bardziej realistyczne modelowanie i ocenę ich zachowania w rzeczywistych warunkach eksploatacyjnych.
W praktyce, losowość może wpływać na:

\begin{itemize}
    \item Bezpieczeństwo i niezawodność konstrukcji: Analiza probabilistyczna uwzględniająca losowość pozwala na oszacowanie ryzyka awarii lub przekroczenia stanów granicznych.
    \item Optymalizacja projektowania: Uwzględnienie odchyłek może prowadzić do bardziej ekonomicznych i efektywnych projektów, które są lepiej dostosowane do rzeczywistych warunków eksploatacyjnych.
    \item Zarządzanie ryzykiem: Analiza wrażliwości, której częścią jest badanie wpływu losowości, pozwala na identyfikację krytycznych parametrów i opracowanie strategii zarządzania ryzykiem.
\end{itemize}

\subsection{Metoda Monte Carlo}

Metoda Monte Carlo jest techniką numeryczną wykorzystywaną do analizy i rozwiązywania problemów, które zawierają elementy losowości lub niepewności.
W kontekście inżynierii, szczególnie przy analizie wrażliwości konstrukcji, metoda Monte Carlo pozwala na ocenę wpływu zmienności parametrów na zachowanie konstrukcji.
Polega na wielokrotnym przeprowadzaniu symulacji z losowo zmienianymi wartościami wejściowymi i analizowaniu wyników w celu zrozumienia, jak różne czynniki wpływają na wynik końcowy, np. na przemieszczenia lub naprężenia w konstrukcji.

\subsubsection*{Jak działa metoda Monte Carlo w analizie wrażliwości konstrukcji?}
\begin{enumerate}
\item Określenie parametrów losowych:

\begin{itemize}
\item W pierwszym kroku identyfikowane są parametry konstrukcji, które mogą podlegać losowości. Mogą to być na przykład moduł sprężystości materiału, geometria elementów, czy wartości sił zewnętrznych. Każdy z tych parametrów jest opisany odpowiednim rozkładem prawdopodobieństwa (np. normalnym, logarytmiczno-normalnym).
\end{itemize}

\item Generowanie prób losowych:

\begin{itemize}
\item Dla każdego parametru generowane są losowe wartości zgodnie z określonymi rozkładami prawdopodobieństwa. Na przykład, jeśli moduł sprężystości materiału jest zmienny w przedziale ±10% od wartości nominalnej, to w każdej symulacji metoda Monte Carlo losowo wybiera wartość modułu sprężystości z tego przedziału.
\end{itemize}

\item Symulacja konstrukcji:

\begin{itemize}
\item Po wygenerowaniu zestawu losowych parametrów, przeprowadza się symulację konstrukcji, np. za pomocą metody elementów skończonych (MES). Symulacja ta dostarcza informacji o odpowiedzi konstrukcji na zadane obciążenia, takich jak przemieszczenia, naprężenia, czy siły wewnętrzne.
\end{itemize}

\item Powtórzenie symulacji:

\begin{itemize}
\item Proces generowania losowych wartości parametrów i symulacji konstrukcji jest powtarzany bardzo wiele razy, np. tysiące lub nawet miliony razy. Każda z tych prób daje jeden wynik dla badanej wielkości, np. przemieszczenia w wybranym węźle.
\end{itemize}

\item Analiza statystyczna wyników:

\begin{itemize}
\item Wyniki wszystkich symulacji są następnie analizowane statystycznie. Można w ten sposób uzyskać rozkład przemieszczeń, wartość średnią, odchylenie standardowe, a także prawdopodobieństwo przekroczenia określonego limitu (np. maksymalne dopuszczalne przemieszczenie).
\end{itemize}

\end{enumerate}
\subsubsection*{Zastosowanie metody Monte Carlo w analizie wrażliwości konstrukcji}
Metoda Monte Carlo jest szczególnie użyteczna w analizie wrażliwości konstrukcji, ponieważ umożliwia:

\begin{itemize}
\item Ocenę wpływu niepewności na wyniki:

\begin{itemize}
\item Możemy ocenić, jak bardzo niepewność w parametrach takich jak moduł sprężystości czy obciążenia wpływa na przemieszczenia lub naprężenia w konstrukcji. Na przykład, możemy stwierdzić, że w 95% przypadków przemieszczenie w danym węźle nie przekroczy określonej wartości.
\end{itemize}

\item Identyfikację kluczowych parametrów:

\begin{itemize}
\item Analiza wyników pozwala zidentyfikować, które z parametrów mają największy wpływ na zmienność wyników. Może to pomóc w podejmowaniu decyzji dotyczących optymalizacji projektu czy potrzebnych badań materiałowych.
\end{itemize}

\item Zarządzanie ryzykiem:

\begin{itemize}
    \item Dzięki metodzie Monte Carlo można przewidzieć i zminimalizować ryzyko awarii konstrukcji, wynikające z losowości parametrów. Na przykład, jeśli określone parametry mają dużą niepewność, można zwiększyć ich marginesy bezpieczeństwa.
\end{itemize}

\end{itemize}

\subsection{Metoda elementów skończonych MES}

Metoda elementów skończonych (MES) jest zaawansowaną techniką obliczeniową stosowaną do analizy i rozwiązywania problemów inżynierskich, w których wymagane jest zbadanie rozkładu naprężeń, przemieszczeń, temperatury czy innych parametrów fizycznych w konstrukcjach i materiałach.
MES polega na dyskretyzacji ciągłej przestrzeni, takiej jak belka, płyta czy bryła, na mniejsze, skończone elementy.
Każdy z tych elementów jest połączony z sąsiadującymi węzłami, które stanowią punkty, w których obliczane są wartości szukanych wielkości, takich jak przemieszczenia czy naprężenia.

Podstawą działania metody elementów skończonych jest podział skomplikowanego problemu na wiele prostszych, łatwiejszych do rozwiązania części.
Cały obiekt inżynierski, na przykład most, jest dzielony na niewielkie, regularne elementy – trójkąty, czworokąty w przypadku problemów płaskich, lub czworościany i sześciany w przypadku problemów przestrzennych.
Każdy z tych elementów jest opisywany przez funkcje aproksymujące, które przybliżają rozkład szukanych wielkości na całym elemencie na podstawie wartości tych wielkości w węzłach elementu.

Kluczowym etapem w MES jest utworzenie macierzy sztywności, która reprezentuje zależności między obciążeniami działającymi na konstrukcję a przemieszczeniami w węzłach.
Macierz ta jest następnie używana do rozwiązania układu równań liniowych, który opisuje stan równowagi całej konstrukcji.
W zależności od liczby elementów oraz ich skomplikowania, układ równań może mieć bardzo dużą liczbę niewiadomych, co wymaga zastosowania odpowiednich algorytmów numerycznych oraz dużej mocy obliczeniowej.

MES jest szczególnie użyteczna w analizie skomplikowanych konstrukcji, dla których rozwiązanie analityczne jest niemożliwe lub bardzo trudne do uzyskania.
Dzięki MES inżynierowie mogą symulować zachowanie konstrukcji pod różnymi obciążeniami i warunkami brzegowymi, co umożliwia optymalizację projektu przed jego realizacją.

\subsubsection*{Równanie MES w statyce}

Równanie podstawowe w metodzie elementów skończonych (MES) w statyce, znane jako równanie równowagi, jest kluczowym narzędziem do analizy konstrukcji inżynierskich. Równanie to jest wyrażeniem matematycznym, które opisuje równowagę sił działających na konstrukcję w stanie spoczynku, bez uwzględniania wpływu czasu.

W najprostszej formie równanie MES w statyce można zapisać jako:

$Ku=f$

gdzie:
\begin{itemize}
    \item \textbf{K} – macierz sztywności,
    \item \textbf{u} – wektor przemieszczeń,
    \item \textbf{f} – wektor sił zewnętrznych.
\end{itemize}

\subsubsection*{Wyjaśnienie poszczególnych składników:}
\begin{enumerate}
\item Macierz sztywności K\mathbf{K}K:
Jest to macierz, która zawiera informacje o właściwościach materiałowych i geometrycznych elementów oraz o ich wzajemnych połączeniach w konstrukcji. Macierz ta jest symetryczna i dodatnio określona, co oznacza, że wszystkie jej elementy są dodatnie lub zero, a jej diagonalne elementy (te na głównej przekątnej) są zawsze dodatnie. Macierz sztywności łączy przemieszczenia w węzłach z odpowiadającymi im siłami reakcji.


\item Wektor przemieszczeń u\mathbf{u}u:
Wektor ten zawiera przemieszczenia w węzłach konstrukcji, które są niewiadomymi w równaniu
\item Każdy element tego wektora odpowiada przemieszczeniu (lub rotacji) w danym węźle konstrukcji, w określonym kierunku (osi X, Y lub Z).


\item Wektor sił zewnętrznych f\mathbf{f}f:
Jest to wektor, który reprezentuje wszystkie zewnętrzne siły i momenty przyłożone do konstrukcji.
\item Wektor ten zawiera siły, które dążą do deformacji konstrukcji, takie jak obciążenia stałe, obciążenia użytkowe, siły wiatru, ciężar własny konstrukcji itp.


\end{enumerate}

\subsubsection*{Interpretacja równania:}
Równanie Ku=f opisuje równowagę sił w konstrukcji.
Oznacza to, że siły wewnętrzne (reprezentowane przez macierz sztywności i przemieszczenia) równoważą siły zewnętrzne.
W praktyce inżynierskiej rozwiązanie tego równania pozwala na obliczenie przemieszczeń w konstrukcji, gdy znane są siły zewnętrzne.

\subsection{Narzędzia}
\subsubsection{Python}

Python to wszechstronny język programowania o otwartym kodzie źródłowym, który cieszy się ogromną popularnością w różnych dziedzinach, w tym w inżynierii i nauce.
Jego prosta składnia, bogata biblioteka standardowa oraz dostępność licznych bibliotek zewnętrznych sprawiają, że Python jest idealnym narzędziem do implementacji algorytmów analizy wrażliwości, przetwarzania danych oraz wizualizacji wyników.
W kontekście niniejszej pracy, Python będzie wykorzystywany do automatyzacji procesu analizy, generowania modeli konstrukcji, przeprowadzania symulacji oraz przetwarzania i prezentacji wyników.

\subsubsection{OpenSees}

OpenSees (Open System for Earthquake Engineering Simulation) to zaawansowana platforma programistyczna o otwartym kodzie źródłowym, stworzona specjalnie do symulacji zachowania konstrukcji pod wpływem obciążeń sejsmicznych i innych dynamicznych.
Oferuje ona szeroki zakres elementów skończonych, modeli materiałowych oraz algorytmów rozwiązywania, co czyni ją doskonałym narzędziem do analizy wrażliwości konstrukcji ramowych.
Integracja OpenSees z Pythonem pozwala na pełną automatyzację procesu analizy, umożliwiając przeprowadzanie złożonych symulacji oraz analizę wyników w sposób efektywny i elastyczny.
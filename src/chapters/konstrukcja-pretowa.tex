Przedmiotem analizy jest konstrukcja składająca się z trzech elementów prętowych, przy czym każdy z prętów jest wykonany
z innego materiału, co oznacza, że mają one różne moduły sprężystości \textbf{E} oraz powierzchnie przekroju poprzecznego \textbf{A}.

Każdy z elementów konstrukcji jest zamocowany u góry za pomocą przegubu nieprzesuwnego, a w węźle 4 elementy są połączone między sobą przegubowo.
Konstrukcja poddana jest działaniu siły skupionej \textbf{P} w węźle 4, skierowanej skośnie w prawy dół.

Analiza tej konstrukcji koncentruje się na ocenie przemieszczenia węzła 4 w kierunku pionowym (w osi Y) w odpowiedzi na działanie siły \textbf{P}.
Celem tych symulacji jest określenie, jak zmiany w parametrach wpływają na przemieszczenie węzła 4 oraz ocena zbieżności wyników przy różnych liczbach prób.

\begin{table}[H]
    \centering
    \begin{tabular}{|c|c|c|c|c|c|c|c|c|c|c|}
        \hline
        $a$ & $b$ & $c$ & $E_1$ & $E_2$ & $E_3$ & $A_1$ & $A_2$ & $A_3$ & $P_x$ & $P_y$ \\
        \hline
        \multicolumn{3}{|c|}{$m$} & \multicolumn{3}{|c|}{$X$} & \multicolumn{3}{|c|}{$m^2$} & \multicolumn{2}{|c|}{$kN$} \\
        \hline
        3 & 4 & 5 & $30*10^6$ & $40*10^6$ & $50*10^6$ & 0.3 & 0.2 & 0.1 & 15 & -5 \\
        \hline
    \end{tabular}
    \caption{Parametry konstrukcji prętowej}
    \label{tab:pars-kp}
\end{table}

\resultstable
{1 & 2 & 3 & 4 & 5 & 6}
{1 & 2 & 3 & 4 & 5 & 6}
{1 & 2 & 3 & 4 & 5 & 6}
{Wyniki symulacji dla konstrukcji prętowej}
{kp-results}

\opsfigure{Konstrukcja_pretowa_model}{Model konstrukji – Konstrukcja prętowa}
\opsfigure{Konstrukcja_pretowa_10000_def}{Deformacja konstrukji – Konstrukcja prętowa 10000}
\opsfigure{Konstrukcja_pretowa_10000_sp}{Przesunięcie węzła 4 w kierunku y – Konstrukcja prętowa 10000}
\opsfigure{Konstrukcja_pretowa_10000_zsp}{Zbieżność wartości średniej przemieszczenia - Konstrukcja prętowa 10000}
\opsfigure{Konstrukcja_pretowa_10000_zos}{Zbieżność odchylenia standardowego – Konstrukcja prętowa 10000}

\opsfigure{Konstrukcja_pretowa_100000_def}{Deformacja konstrukji – Konstrukcja prętowa 100000}
\opsfigure{Konstrukcja_pretowa_100000_sp}{Przesunięcie węzła 4 w kierunku y – Konstrukcja prętowa 100000}
\opsfigure{Konstrukcja_pretowa_100000_zsp}{Zbieżność wartości średniej przemieszczenia - Konstrukcja prętowa 100000}
\opsfigure{Konstrukcja_pretowa_100000_zos}{Zbieżność odchylenia standardowego – Konstrukcja prętowa 100000}

\opsfigure{Konstrukcja_pretowa_1000000_def}{Deformacja konstrukji – Konstrukcja prętowa 1000000}
\opsfigure{Konstrukcja_pretowa_1000000_sp}{Przesunięcie węzła 4 w kierunku y – Konstrukcja prętowa 1000000}
\opsfigure{Konstrukcja_pretowa_1000000_zsp}{Zbieżność wartości średniej przemieszczenia - Konstrukcja prętowa 1000000}
\opsfigure{Konstrukcja_pretowa_1000000_zos}{Zbieżność odchylenia standardowego – Konstrukcja prętowa 1000000}

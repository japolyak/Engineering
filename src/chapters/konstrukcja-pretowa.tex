Przedmiotem analizy jest konstrukcja składająca się z trzech elementów prętowych, przy czym każdy z prętów jest wykonany
z innego materiału, co oznacza, że mają one różne moduły sprężystości \textbf{E} oraz powierzchnie przekroju poprzecznego \textbf{A}.

Każdy z elementów konstrukcji jest zamocowany u góry za pomocą przegubu nieprzesuwnego, a w węźle 4 elementy są połączone między sobą przegubowo.

Konstrukcja poddana jest działaniu siły skupionej \textbf{P} przyłożonej w węźle 4, skierowanej skośnie w prawy dół.
W implementacji siłę \textbf{P} rozłożono na składowe w płaszczyźnie Y oraz X.

Analiza skupia się na ocenie przemieszczenia węzła 4 w kierunku pionowym (w osi Y) pod wpływem działania siły \textbf{P}.

\cadmodel
{Prentowa-mod}
{Model konstrukcji prętowej}

\begin{table}[H]
    \centering
    \begin{tabular}{|c|c|c|c|c|c|c|c|c|c|c|}
        \hline
        $a$ & $b$ & $c$ & $E_1$ & $E_2$ & $E_3$ & $A_1$ & $A_2$ & $A_3$ & $P_x$ & $P_y$ \\
        \hline
        \multicolumn{3}{|c|}{$m$} & \multicolumn{3}{|c|}{$X$} & \multicolumn{3}{|c|}{$m^2$} & \multicolumn{2}{|c|}{$kN$} \\
        \hline
        3 & 4 & 5 & $30*10^6$ & $40*10^6$ & $50*10^6$ & 0.3 & 0.2 & 0.1 & 15 & -5 \\
        \hline
    \end{tabular}
    \caption{Parametry konstrukcji prętowej}
    \label{tab:pars-kp}
\end{table}

\constructionresults
{\prentowa}
{Konstrukcja_pretowa}
{Przesunięcie węzła 4}
{
\resultstable
{4.24 & -1.16 & 3.34 & 0.15 & 2.11 & 4.35}
{10.8 & -5.96 & 3.53 & 0.16 & 2.59 & 4.56}
{32.9 & -5.9 & 3.5 & 0.16 & 2.67 & 4.68}
{Wyniki symulacji dla konstrukcji prętowej}
{kp-results}
{$[m] 10^{-3}$ & $[m] 10^{-3}$ & $[m] 10^{-5}$ & $[m] 10^{-3}$ & $10^{-8}$ & $\%$}
}

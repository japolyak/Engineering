Rozpatrywana jest belka swobodnie podparta, składająca się z trzech elementów, wykonanych z jednorodnego materiału o jednakowych właściwościach fizycznych i geometrycznych:
moduł sprężystości \textbf{E}, moment bezwładności przekroju \textbf{I} oraz powierzchnia przekroju poprzecznego \textbf{A}.
Belka jest podparta na dwóch podporach przegubowo nieprzesuwnych oraz jednej podporze przegubowo przesuwnej.

Konstrukcję poddano działaniu dwóch rodzajów obciążeń:

\begin{itemize}
    \item Obciążenia równomiernie rozłożonego $q$ na całej długości pierwszego elementu, skierowanego pionowo w dół.
    \item Obciążeniea punktowego $P$ przyłożonego na końcu belki (węzeł 4), skierowanego pionowo w dół.
\end{itemize}

Analiza skupia się na przemieszczeniu węzła 4 w kierunku pionowym (w osi Y).

\cadmodel
{Belka-mod}
{Model belki swobodnie podpartej}
{0.35}

\begin{table}[H]
    \centering
    \begin{tabular}{|c|c|c|c|c|c|c|c|c|}
        \hline
        Rodzaj parametru & L & b & d & A & $\mathrm{I}_\mathrm{z}$ & E & P & q \\
        \hline
        – & \multicolumn{3}{|c|}{[m]} & [$\mathrm{m}^\mathrm{2}$] & [$\mathrm{m}^\mathrm{4}$] & [GPa] & [kN] & [kN/m] \\
        \hline
        Deterministyczny & 3 & 0.3 & 0.6 & 0.18 & – & – & – & – \\
        \hline
        Średni & – & – & – & – & 145.8 \cdot $\mathrm{10}^{\mathrm{-6}}$ & 30 & –10 & –5 \\
        \hline
        Odchylenie standardowe & – & – & – & – & 14.6 \cdot $\mathrm{10}^{\mathrm{-6}}$ & 3 & 1 & 0.5 \\
        \hline
    \end{tabular}
    \caption{Parametry belki swobodnie podpartej}
    \label{tab:pars-belka}
\end{table}

\constructionresults
{\belka}
{Belka}
{Ugięcie na końcu belki}
{
\resultstable
{-65.5 \cdot $\mathrm{10}^{\mathrm{-8}}$ & -40430 & -36.5 & 674.1 & 454357.3 & -18.46}
{-48 \cdot $\mathrm{10}^{\mathrm{-8}}$ & -556007 & -45.2 & 2146.4 & 4607043 & -47.6}
{-2.32 \cdot $\mathrm{10}^{\mathrm{-8}}$ & -832834.6 & -37.9 & 1851.5 & 3428174.8 & -48.88}
{Wyniki symulacji dla belki}
{belka-results}
{[m] & [m] & [m] & [m] & [m] & [\%]}
}
{1}
{1}
{1}
{1}

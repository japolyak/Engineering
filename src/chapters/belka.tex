Rozpatrywana jest belka swobodnie podparta, składająca się z trzech elementów, wykonanych z jednorodnego materiału o jednakowych właściwościach fizycznych i geometrycznych:
moduł sprężystości \textbf{E}, moment bezwładności przekroju \textbf{I} oraz powierzchnia przekroju poprzecznego \textbf{A}.
Belka jest podparta na dwóch podporach przegubowo nieprzesuwnych oraz jednej podporze przegubowo przesuwnej.

Konstrukcję poddano działaniu dwóch rodzajów obciążeń:

\begin{itemize}
    \item Obciążenia równomiernie rozłożonego q na całej długości pierwszego elementu, skierowanego pionowo w dół.
    \item Obciążeniea punktowego P przyłożonego na końcu belki (węzeł 4), skierowanego pionowo w dół.
\end{itemize}

Na Rysunku \ref{cadmodel:Belka-mod} jest przedstawiony model belki swobodnie podpartej, a Tabela \ref{tab:parametry-belka} zawiera deterministyczne oraz losowane parametry.
Implementacja obliczeń jest zawarta w załączniku \ref{appendix:Belka}.

Analiza skupia się na przemieszczeniu węzła 4 w kierunku pionowym (w osi Y).

\cadmodel
{Belka-mod}
{Model belki swobodnie podpartej}
{0.35}

\begin{table}[H]
    \centering
    \begin{tabular}{|c|c|c|c|c|c|c|c|c|}
        \hline
        Rodzaj parametru & L & b & d & A & $\mathrm{I}_\mathrm{z}$ & E & P & q \\
        \hline
        – & \multicolumn{3}{|c|}{[m]} & [$\mathrm{m}^\mathrm{2}$] & [$\mathrm{m}^\mathrm{4}$] & [GPa] & [kN] & [kN/m] \\
        \hline
        Deterministyczny & 3 & 0.3 & 0.6 & 0.18 & – & – & – & – \\
        \hline
        Średni & – & – & – & – & 145.8 \cdot $\mathrm{10}^{\mathrm{-6}}$ & 30 & –10 & –5 \\
        \hline
        Odchylenie standardowe & – & – & – & – & 14.6 \cdot $\mathrm{10}^{\mathrm{-6}}$ & 3 & 1 & 0.5 \\
        \hline
    \end{tabular}
    \caption{Parametry belki swobodnie podpartej}
    \label{tab:parametry-belka}
\end{table}

\newpage
\constructionresults
{\belka}
{Belka}
{Ugięcie na końcu belki}
{
\resultstable
{-0.66 & -40.43 & -0.037 & 0.67 & 0.45 & -18.46}
{-0.48 & -556 & -0.045 & 2.15 & 4.61 & -47.6}
{-0.02 & -832.8 & -0.038 & 1.85 & 3.43 & -48.88}
{Wyniki symulacji dla belki}
{Belka-results}
{[μm] & [mm] & [mm] & [mm] & [$\mathrm{mm}^\mathrm{2}$] & [\%]}
}
{
Rysunek \ref{fig:Belka_model} przedstawia model obliczeniowy belki swobodnie podpartej wygenerowany w środowisku symulacyjnym.
Na Rysunku \ref{fig:Belka_deformation} ukazano deformację belki pod wpływem siły skupionej pionowej.
Tabela \ref{tab:Kratownica-results} zawiera wartości przemieszczeń uzyskane w 3 seriach obliczeń.

Warto zauważyć, że wraz ze wzrostem liczby prób symulacyjnych, różnica między maksymalnym a minimalnym przemieszczeniem wzrasta, mimo że dla każdej serii pomiarów jest ona zbyt wielka.
}
{
Rysunki \ref{fig:Belka_10000_sp}, \ref{fig:Belka_10000_zsp} oraz \ref{fig:Belka_10000_zos} prezentują wyniki przemieszczenia węzła 4 dla 10.000 symulacji.
Średnie przemieszczenie wynosi 0.037 mm, a różnica między maksymalną wartością (-0.66 μm) a minimalną (-40.43 mm) wynosi prawie tyle samo co wartość minimalna.
Na Rysunku \ref{fig:Belka_10000_sp} widoczne są przemieszczenia znacząco odbiegające od średniej, co może świadczyć o występowaniu zaburzeń.
Z kolei na Rysunkach \ref{fig:Belka_10000_zsp} – \ref{fig:Belka_10000_zos} widać, że do stabilizacji wyników jeszcze nie doszło.
}
{
Wyniki dla 100.000 symulacji przedstawiono na Rysunkach \ref{fig:Belka_100000_sp}, \ref{fig:Belka_100000_zsp} oraz \ref{fig:Belka_100000_zos}.
Średnie przemieszczenie zwiększyło się do 0.045 mm, a różnica między wartością maksymalną (-0.48 μm) a minimalną (-556 mm) nadal jest zbyt wielka.
Na Rysunku \ref{fig:Belka_100000_sp} nadal obserwowane są przemieszczenia, które mogą być uznane za zaburzenia.
Stabilizacje średniej wartości przemieszczenia oraz odchylenia standardowego (Rysunek \ref{fig:Belka_100000_zsp} – \ref{fig:Belka_100000_zos}) nie są jezcze widoczne.
}
{
Dla 1.000.000 symulacji, wyniki na  Rysunkach \ref{fig:Belka_1000000_sp}, \ref{fig:Belka_1000000_zsp} oraz \ref{fig:Belka_1000000_zos} świadczą, że średnie przemieszczenie wynosi 0.038 mm, a różnica między wartością maksymalną (-0.02 μm) a minimalną (-828.8 mm) jescze bardziej wzrosła.
Na Rysunku \ref{fig:Belka_1000000_sp} wciąż widoczne są wartości przemieszczeń, które mogą być traktowane jako zaburzenia.
Całkowitą stabilizację średniej wartości przemieszczenia można zaobserwować po około 800.000 symulacji (Rysunek \ref{fig:Belka_1000000_zsp}),
natomiast stabilizacja odchylenia standardowego niby następuje dopiero po około 1.00.000 prób (Rysunek \ref{fig:Belka_1000000_zos}).
}

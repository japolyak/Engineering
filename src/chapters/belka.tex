Rozpatrywana jest belka swobodnie podparta, składająca się z trzech elementów, wykonanych z jednorodnego materiału o jednakowych właściwościach fizycznych i geometrycznych:
moduł sprężystości \textbf{E}, moment bezwładności przekroju \textbf{I} oraz powierzchnia przekroju poprzecznego \textbf{A}.
Belka jest podparta na dwóch podporach przegubowo nieprzesuwnych oraz jednej podporze przegubowo przesuwnej.

Konstrukcję poddano działaniu dwóch rodzajów obciążeń:

\begin{itemize}
    \item Obciążenia równomiernie rozłożonego $q$ na całej długości pierwszego elementu, skierowanego pionowo w dół.
    \item Obciążeniea punktowego $P$ przyłożonego na końcu belki (węzeł 4), skierowanego pionowo w dół.
\end{itemize}

Analiza skupia się na przemieszczeniu węzła 4 w kierunku pionowym (w osi Y).

\cadmodel
{Belka-mod}
{Model belki swobodnie podpartej}

\begin{table}[H]
    \centering
    \begin{tabular}{|c|c|c|c|c|c|c|c|}
        \hline
        $L$ & $b$ & $d$ & $A$ & $I_z$ & $E$ & $P$ & $q$ \\
        \hline
        \multicolumn{3}{|c|}{$m$} & $m^2$ & $m^4$ & $X$ & $kN$ & $kN/m$ \\
        \hline
        3 & 0.3 & 0.6 & 0.18 & $145.8*10^-6$ & $30*10^6$ & -10 & -5 \\
        \hline
    \end{tabular}
    \caption{Parametry belki}
    \label{tab:pars-belka}
\end{table}

\constructionresults
{\belka}
{Belka}
{Ugięcie na końcu belki}
{
\resultstable
{-277.7 & -64598.1 & -34.56 & 822.92 & 677194.25 & -23.81}
{-21.1 & -304406.85 & -33.43 & 1216.47 & 1479801.34 & -36.39}
{-0.02 & -932248.17 & -34.55 & 1686.36 & 2843799.23 & -48.81}
{Wyniki symulacji dla belki}
{belka-results}
{$10^{-8}$ & $uMin$ & $uMean$ & $uStd$ & $uVar$ & $uCov$}
}

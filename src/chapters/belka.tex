rozpatrywana jest belka swobodnie podparta, składająca się z trzech elementów, wykonanych z jednorodnego materiału o jednakowych właściwościach fizycznych i geometrycznych,
takich jak moduł sprężystości \textbf{E} moment bezwładności przekroju \textbf{I}  oraz powierzchnia przekroju poprzecznego \textbf{A}.

Belka jest podparta na trzech podporach: dwóch podporach przegubowo nieprzesuwnych i jednej przegubowo przesuwnej.
Konstrukcja poddana jest działaniu dwóch rodzajów obciążeń:

\begin{itemize}
    \item Siła ciągła $q$, równomiernie rozłożona na całej długości pierwszego elementu, skierowana pionowo w dół.
    \item Siła skupiona $P$ przyłożona na końcu belki (węzeł 4), skierowana pionowo w dół.
\end{itemize}

Analiza skupia się na przemieszczeniu węzła 4 w kierunku pionowym (w osi Y). W badaniach analizowano,
jak te obciążenia wpływają na przemieszczenie węzła oraz jak wyniki konwergują przy różnych liczbach prób.

\begin{table}[H]
    \centering
    \begin{tabular}{|c|c|c|c|c|c|c|c|}
        \hline
        $L$ & $b$ & $d$ & $A$ & $I_z$ & $E$ & $P$ & $q$ \\
        \hline
        \multicolumn{3}{|c|}{$m$} & $m^2$ & $m^4$ & $X$ & $kN$ & $kN/m$ \\
        \hline
        3 & 0.3 & 0.6 & 0.18 & $145.8*10^-6$ & $30*10^6$ & -10 & -5 \\
        \hline
    \end{tabular}
    \caption{Parametry belki}
    \label{tab:pars-belka}
\end{table}

\resultstable
{1 & 2 & 3 & 4 & 5 & 6}
{1 & 2 & 3 & 4 & 5 & 6}
{1 & 2 & 3 & 4 & 5 & 6}
{Wyniki symulacji dla belki}
{belka-results}

\opsfigure{Belka_model}{Model konstrukji – Belka}
\opsfigure{Belka_deformation}{Deformacja konstrukji – Belka}

\opsfigure{Belka_10000_sp}{Ugięcie na końcu belki – Belka 10000}
\opsfigure{Belka_10000_zsp}{Zbieżność wartości średniej przemieszczenia - Belka 10000}
\opsfigure{Belka_10000_zos}{Zbieżność odchylenia standardowego – Belka 10000}

\opsfigure{Belka_100000_sp}{Ugięcie na końcu belki – Belka 100000}
\opsfigure{Belka_100000_zsp}{Zbieżność wartości średniej przemieszczenia - Belka 100000}
\opsfigure{Belka_100000_zos}{Zbieżność odchylenia standardowego – Belka 100000}

\opsfigure{Belka_1000000_sp}{Ugięcie na końcu belki – Belka 1000000}
\opsfigure{Belka_1000000_zsp}{Zbieżność wartości średniej przemieszczenia - Belka 1000000}
\opsfigure{Belka_1000000_zos}{Zbieżność odchylenia standardowego – Belka 1000000}

Przedmiotem analizy jest kratownicowa składająca się z trzech elementów prętowych, przy czym każdy z prętów jest wykonany
z innego materiału, co oznacza, że mają one różne moduły sprężystości \textbf{E} oraz powierzchnie przekroju poprzecznego \textbf{A}.

Każdy z elementów kratownicy jest zamocowany u góry za pomocą przegubu nieprzesuwnego, a w węźle 4 elementy są połączone między sobą przegubowo.

Konstrukcja poddana jest działaniu siły skupionej \textbf{P} przyłożonej w węźle 4, skierowanej skośnie w prawy dół.
W implementacji siłę \textbf{P} rozłożono na składowe w płaszczyźnie Y oraz X.

Na Rysunku \ref{cadmodel:Kratownica-mod} jest przedstawiony model kratownicy, a Tabela \ref{tab:parametry-kratownica} zawiera deterministyczne oraz losowane parametry konstrukcji.
Implementacja obliczeń jest zawarta w załączniku \ref{appendix:Kratownica}.

Analiza skupia się na ocenie przemieszczenia węzła 4 w kierunku pionowym (w osi Y) pod wpływem działania siły \textbf{P}.

\cadmodel
{Kratownica-mod}
{Model kratownicy}
{0.35}

\begin{table}[H]
    \centering
    \begin{tabular}{|c|c|c|c|c|c|c|c|c|c|c|c|}
        \hline
        Rodzaj parametru & a & b & c & $\mathrm{E}_\mathrm{1}$ & $\mathrm{E}_\mathrm{2}$ & $\mathrm{E}_\mathrm{3}$ & $\mathrm{A}_\mathrm{1}$ & $\mathrm{A}_\mathrm{2}$ & $\mathrm{A}_\mathrm{3}$ & $\mathrm{P}_\mathrm{x}$ & $\mathrm{P}_\mathrm{y}$ \\
        \hline
        – & \multicolumn{3}{|c|}{[m]} & \multicolumn{3}{|c|}{[GPa]} & \multicolumn{3}{|c|}{[$\mathrm{m}^\mathrm{2}$]} & \multicolumn{2}{|c|}{[kN]} \\
        \hline
        Deterministyczny & 3 & 4 & 5 & 30 & 40 & 50 & – & – & – & – & – \\
        \hline
        Średni & – & – & – & – & – & – & 0.3 & 0.2 & 0.1 & 15 & –5 \\
        \hline
        Odchylenie standardowe & – & – & – & – & – & – & 0.03 & 0.02 & 0.01 & 1.5 & 0.5 \\
        \hline
    \end{tabular}
    \caption{Parametry kratownicy}
    \label{tab:parametry-kratownica}
\end{table}

\constructionresults
{\kratownica}
{Kratownica}
{Przesunięcie węzła 4}
{
\resultstable
{3.78 & -1.06 & 0.036 & 0.15 & 2.35 & 4.3}
{8.8 & -2.74 & 0.036 & 0.16 & 24.74 & 4.4}
{16.1 & -6 & 0.035 & 0.16 & 24.4 & 4.46}
{Wyniki symulacji dla kratownicy}
{Kratownica-results}
{[mm] & [mm] & [mm] & [mm] & [nm] & [\%]}
}
{
Rysunek \ref{fig:Kratownica_model} przedstawia model obliczeniowy konstrukcji kratownicowej wygenerowany w środowisku symulacyjnym.
Na Rysunku \ref{fig:Kratownica_deformation} ukazano deformację kratownicy pod wpływem siły skośnej, przyłożonej w węźle 4.
Tabela \ref{tab:Kratownica-results} zawiera wartości przemieszczeń uzyskane w 3 seriach obliczeń.

Warto zauważyć, że wraz ze wzrostem liczby prób symulacyjnych, różnica między maksymalnym a minimalnym przemieszczeniem wzrasta.
Ponadto, stosunek minimalnej wartości przemieszczenia, najbardziej zbliżonej do wartości średniej, do samej wartości średniej rośnie z 30 do 170 w miarę zwiększania liczby symulacji.
}
{
Rysunki \ref{fig:Kratownica_10000_sp}, \ref{fig:Kratownica_10000_zsp} oraz \ref{fig:Kratownica_10000_zos} prezentują wyniki przemieszczenia węzła 4 dla 10.000 symulacji.
Średnie przemieszczenie wynosi 0.036 mm, a różnica między maksymalną wartością (3.78 mm) a minimalną (-1.06 mm) wynosi 4.84 mm.
Na Rysunku \ref{fig:Kratownica_10000_sp} widoczne są przemieszczenia znacząco odbiegające od średniej, co może świadczyć o występowaniu zaburzeń.
Z kolei na Rysunkach \ref{fig:Kratownica_10000_zsp} – \ref{fig:Kratownica_10000_zos} widać, że stabilizacja wyników jeszcze nie nastąpiła.
}
{
Wyniki dla 100.000 symulacji przedstawiono na Rysunkach \ref{fig:Kratownica_100000_sp}, \ref{fig:Kratownica_100000_zsp} oraz \ref{fig:Kratownica_100000_zos}.
Średnie przemieszczenie wynosi 0.036 mm, a różnica między wartością maksymalną (8.8 mm) a minimalną (-2.74 mm) wzrosła do 11.54 mm.
Na Rysunku \ref{fig:Kratownica_100000_sp} nadal obserwowane są pojedyncze przemieszczenia, które mogą być uznane za zaburzenia.
Stabilizacja średniej wartości przemieszczenia jest zauważalna po około 60.000 próbach (Rysunek \ref{fig:Kratownica_100000_zsp}),
 jednak stabilizacja odchylenia standardowego (Rysunek \ref{fig:Kratownica_100000_zos}) nie jest jezcze widoczna.
}
{
Dla 1.000.000 symulacji, wyniki na  Rysunkach \ref{fig:Kratownica_1000000_sp}, \ref{fig:Kratownica_1000000_zsp} oraz \ref{fig:Kratownica_1000000_zos} świadczą, że średnie przemieszczenie wynosi 0.035 mm, a różnica między wartością maksymalną (16.1 mm) a minimalną (-6 mm) to 22.1 mm.
Na Rysunku \ref{fig:Kratownica_1000000_sp} wciąż widoczne są wartości przemieszczeń, które mogą być traktowane jako zaburzenia.
Całkowitą stabilizację średniej wartości przemieszczenia można zaobserwować po około 100.000 symulacji (Rysunek \ref{fig:Kratownica_1000000_zsp}),
natomiast stabilizacja odchylenia standardowego następuje dopiero po około 900.000 prób, co jest widoczne na Rysunku \ref{fig:Kratownica_1000000_zos}.
}
\section{Podsumowanie i wnioski}

Celem niniejszej pracy inżynierskiej było przeprowadzenie analizy wrażliwości konstrukcji prętowych, uwzględniając losowość geometrii i właściwości materiałowych.
W ramach pracy szczegółowo omówiono teoretyczne podstawy analizy wrażliwości oraz metody numeryczne, takie jak metoda elementów skończonych (MES) i metoda Monte Carlo, wykorzystane do przeprowadzenia symulacji.

Opracowano modele analizowanych konstrukcji, a następnie, korzystając z silnika obliczeniowego OpenSees, stworzono dedykowane skrypty obliczeniowe.
Skrypty te obejmują definicję środowiska obliczeniowego, modelowanie węzłów, elementów, podpór oraz obciążeń, jak również parametryzację modelu.
Przeprowadzono symulacje dla każdej konstrukcji, wykorzystując 10.000, 100.000 oraz 1.000.000 prób, a uzyskane wyniki zostały zebrane w tabelach i zilustrowane na wykresach.

Analiza wyników dla belki wskazuje na dużą zmienność przemieszczeń przy różnych liczbach prób, ale średnia wartość przemieszczenia stabilizuje się przy większej liczbie symulacji, zwłaszcza po 1.000.000 prób.
Wzrost odchylenia standardowego wraz z liczbą prób jest naturalny, ale stabilizacja średniej wartości oznacza osiągnięcie wiarygodnych wyników.

Wyniki dla ramy przestrzennej pokazują mniejszą zmienność przemieszczeń niż w przypadku innych konstrukcji.
Stabilizacja średniej wartości przemieszczenia następuje już po około 10.000 prób, co świadczy o jej niskiej wrażliwości na zmienność parametrów.
Odchylenie standardowe także szybko stabilizuje się.

Kratownica wykazuje największą zmienność przemieszczeń, a stabilizacja wyników jest trudniejsza i wymaga większej liczby prób.
Duże różnice w przemieszczeniach i wysokie odchylenie standardowe wskazują na wysoką wrażliwość tej konstrukcji na zmienność parametrów i lokalne zmiany w modelu symulacyjnym.

Stabilizacja wyników w metodzie Monte Carlo zależy od rodzaju konstrukcji.
Kratownica, jako konstrukcja bardziej wrażliwa na zmienność parametrów, wymaga większej liczby prób, podczas gdy rama przestrzenna szybko osiąga stabilizację.
Belka również stabilizuje wyniki, ale dopiero przy większej liczbie prób.

Złożone konstrukcje, jak kratownica, potrzebują więcej prób, aby uzyskać stabilne wyniki, podczas gdy prostsze konstrukcje, jak rama, osiągają stabilizację przy mniejszej liczbie symulacji.
Warto te różnice uwzględnić w planowaniu symulacji, aby efektywnie wykorzystać zasoby obliczeniowe i zapewnić dokładność wyników.

Analiza wrażliwości konstrukcji stanowi nieodłączny element procesu projektowania inżynierskiego, niezależnie od skali i złożoności obiektu.
Choć na rynku dostępne są liczne komercyjne narzędzia do analizy konstrukcji, silnik obliczeniowy OpenSees wyróżnia się jako darmowe rozwiązanie, umożliwiające zarówno doświadczonym inżynierom, jak i początkującym projektantom, przeprowadzanie zaawansowanych analiz.
Wykorzystanie OpenSees pozwala na znaczną redukcję kosztów oraz czasu poświęconego na projektowanie, przyczyniając się do zwiększenia efektywności procesu projektowego.

Niniejsza praca pokazuje, że analiza wrażliwości z wykorzystaniem metody Monte Carlo oraz silnika OpenSees stanowi skuteczne narzędzie do oceny wpływu losowości parametrów na zachowanie konstrukcji prętowych.
Uzyskane wyniki mogą być wykorzystane do optymalizacji projektów, zwiększenia ich bezpieczeństwa oraz efektywności, a także do podejmowania świadomych decyzji projektowych w warunkach niepewności.

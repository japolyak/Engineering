\newpage
\section{Podsumowanie}

Celem pracy inżynierskiej było wykonanie analizy wrażliwości konstrukcji ramowych z uwzględnieniem losowości w geometrii i właściwościach materiałowych.
Praca obejmuje szczegółową część merytoryczną dotyczącą przeprowadzenia analiz wrażliwości konstrukcji z wykorzystaniem metody elementów skończonych (MES).
Zostały opisane użyte narzędzia oraz szczegółowy proces tworzenia programu do analizy wybranych konstrukcji.

Po opracowaniu modeli analizowanych konstrukcji oraz zapoznaniu się z dokumentacją techniczną silnika obliczeniowego \textbf{OpenSees}, stworzono dedykowane skrypty obliczeniowe dla każdej z konstrukcji.
Struktura każdego programu obejmuje definicję środowiska obliczeniowego, modelowanie węzłów, elementów, podpór oraz obciążeń, a także parametryzację modelu.
Na koniec wykonywane są obliczenia oraz zapisywane są wyniki analizy.
Dla każdej konstrukcji przeprowadzono symulacje z użyciem 10,000, 100,000 oraz 1,000,000 prób.
Uzyskane wyniki zostały zebrane w tabelach i zilustrowane na wykresach.

Analiza wrażliwości konstrukcji jest kluczowym elementem procesu projektowania inżynierskiego, niezależnie od skali obiektu.
Rynek oferuje wiele płatnych narzędzi, które umożliwiają szybkie i efektywne przeprowadzanie analiz konstrukcji.
Jednakże, silnik obliczeniowy \textbf{OpenSees} jest darmowy i zapewnia projektantom, na każdym etapie kariery, możliwość wykonywania zaawansowanych analiz konstrukcyjnych, co pozwala na znaczną oszczędność kosztów oraz czasu poświęconego na projektowanie.


\section{Podsumowanie}

Celem niniejszej pracy inżynierskiej było przeprowadzenie analizy wrażliwości konstrukcji ramowych, uwzględniając losowość geometrii i właściwości materiałowych.
W ramach pracy szczegółowo omówiono teoretyczne podstawy analizy wrażliwości oraz metody numeryczne, takie jak metoda elementów skończonych (MES) i metoda Monte Carlo, wykorzystane do przeprowadzenia symulacji.

Opracowano modele analizowanych konstrukcji, a następnie, korzystając z silnika obliczeniowego OpenSees, stworzono dedykowane skrypty obliczeniowe.
Skrypty te obejmują definicję środowiska obliczeniowego, modelowanie węzłów, elementów, podpór oraz obciążeń, jak również parametryzację modelu.
Przeprowadzono symulacje dla każdej konstrukcji, wykorzystując 10 000, 100 000 oraz 1 000 000 prób, a uzyskane wyniki zostały zebrane w tabelach i zilustrowane na wykresach.

Analiza wrażliwości konstrukcji stanowi nieodłączny element procesu projektowania inżynierskiego, niezależnie od skali i złożoności obiektu.
Choć na rynku dostępne są liczne komercyjne narzędzia do analizy konstrukcji, silnik obliczeniowy OpenSees wyróżnia się jako darmowe rozwiązanie, umożliwiające zarówno doświadczonym inżynierom, jak i początkującym projektantom, przeprowadzanie zaawansowanych analiz.
Wykorzystanie OpenSees pozwala na znaczną redukcję kosztów oraz czasu poświęconego na projektowanie, przyczyniając się do zwiększenia efektywności procesu projektowego.

Niniejsza praca pokazuje, że analiza wrażliwości z wykorzystaniem metody Monte Carlo oraz silnika OpenSees stanowi skuteczne narzędzie do oceny wpływu losowości parametrów na zachowanie konstrukcji ramowych.
Uzyskane wyniki mogą być wykorzystane do optymalizacji projektów, zwiększenia ich bezpieczeństwa oraz efektywności, a także do podejmowania świadomych decyzji projektowych w warunkach niepewności.
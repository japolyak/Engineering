\newpage
\section{Modelowanie obliczeniowe konstrukcji}

\subsection{Modelowanie układów ramowych}

\subsection{Implementacja obliczeń}
\subsubsection{Konfiguracja środowiska}

Aby móc zacząć pracę z silnikiem obliczeniowym niezbędne jest prawidłowe skonfigurowanie środowiska programistycznego.
%TODO - add hiperlinks
Ze względu na to, iż komputery nie mają wbudowanego języka Python, ów należy pobrać z oficjalnej strony[1].
Postanowiono pobrać Python wersji 3.11, ponieważ biblioteka \textbf{OpenSeesPy} w wersji 3.5.1.12 akurat tego wymaga,
oraz darmowy edytor kodu źródłowego – \textbf{Visual Studio Code}[2], dla wygody napisana kodu.

Po udanej instalacji Pythona stworzono katalog projektu.
Następnie, korzystając z wiersza poleceń, wybrano wcześniej stworzony katalog i wykonano następne polecenia:

\begin{lstlisting}
python3 -m venv venv
source venv/bin/activate
\end{lstlisting}

W wyniku w wybranym katalogu powstało wirtualne środowisko o nazwie \textbf{venv} w celu odizolować zależności projektu.
Następnie dodano pliki o rozszerzeniu \textbf{.py} dla każdej konstrukcji oraz jeden plik pomocniczy, zawierający wspólny
dla każdego procesu obliczeniowego kod.
Na tym etapie kończy się konfiguracja środowiska.

\subsubsection{Instalacja i import modułów}

Korzystając z wiersza poleceń i komendy \inlinecode{pip install <module name>} zainstalowano następujące biblioteki:

\begin{itemize}
    \item OpenSeesPy
    \item OpsVis
    \item SciPy
    \item NumPy
    \item MatPlotLib
\end{itemize}

Po udanej instalacji przy pomocy komendy \textbf{import} zaimportowano moduły do każdego pliku:

\begin{lstlisting}
# Plik shared.py
import numpy as np
import matplotlib.pyplot as plt
\end{lstlisting}

\begin{lstlisting}
# Pliki belka.py, kratownica.py, rama.py, konstrukcja_pretowa.py
import numpy as np
import opsvis as opsv
from scipy.stats import norm
import openseespy.opensees as ops
from shared import show_results
\end{lstlisting}

\subsubsection{Tworzenie modelu obliczeniowego}

Definicję modelu obliczeniowego wykonano poprzez wywołanie komendy
\inlinecode{model('basic', '-ndm', ndm, '-ndf', ndf)}, w którą po kolei przekazano następujące parametry:

\begin{itemize}
    \item liczbę wymiarów modelu \textbf{ndm} (1, 2, 3)
    \item liczbę stopni swobody na węźle \textbf{ndf}
\end{itemize}

Ponieważ analizowano układy dwuwymiarowe, wartość parametru \textbf{ndm} została ustawiona na 2.
Natomiast wartość \textbf{ndf} różniła się w zależności od typu konstrukcji:
\begin{itemize}
    \item kratownica i konstrukcja prętowa – ustawiono na 2 (dwa stopnie translacyjne)
    \item belka i rama przestrzenna – ustawiono na 3 (dwa stopnie translacyjne oraz 1 stopień rotacyjny)
\end{itemize}
Modele kratownicy i konstrukcji prętowej zdefiniowano jako:
\begin{lstlisting}
ops.model('basic', '-ndm', 2, '-ndf', 2)
\end{lstlisting}
Modele belki i ramy przestrzennej zdefiniowano jako:
\begin{lstlisting}
ops.model('basic', '-ndm', 2, '-ndf', 3)
\end{lstlisting}

\subsubsection{Definiowanie parametrów}

\begin{table}[h]
    \centering
    \begin{tabular}{|c|c|c|c|c|c|c|c|c|c|c|}
        \hline
        $a$ & $b$ & $c$ & $E_1$ & $E_2$ & $E_3$ & $A_1$ & $A_2$ & $A_3$ & $P_x$ & $P_y$ \\
        \hline
        \multicolumn{3}{|c|}{$m$} & \multicolumn{3}{|c|}{$X$} & \multicolumn{3}{|c|}{$m^2$} & \multicolumn{2}{|c|}{$kN$} \\
        \hline
        3 & 4 & 5 & $30*10^6$ & $40*10^6$ & $50*10^6$ & 0.3 & 0.2 & 0.1 & 15 & -5 \\
        \hline
    \end{tabular}
    \caption{Parametry konstrukcji prętowej}
\end{table}

\begin{table}[h]
    \centering
    \begin{tabular}{|c|c|c|c|c|c|c|c|c|}
        \hline
        $L$ & $H$ & $b$ & $d$ & $A$ & $I_z$ & $E$ & $P_1$ & $P_2$ \\
        \hline
        \multicolumn{4}{|c|}{$m$} & $m^2$ & $m^4$ & $X$ & \multicolumn{2}{|c|}{$kN$} \\
        \hline
        3 & 4 & 0.3 & 0.6 & 0.18 & $145.8*10^-6$ & $30*10^6$ & 150 & -50 \\
        \hline
    \end{tabular}
    \caption{Parametry ramy}
\end{table}

\begin{table}[h]
    \centering
    \begin{tabular}{|c|c|c|c|c|c|c|c|}
        \hline
        $L$ & $b$ & $d$ & $A$ & $I_z$ & $E$ & $P$ & $q$ \\
        \hline
        \multicolumn{3}{|c|}{$m$} & $m^2$ & $m^4$ & $X$ & $kN$ & $kN/m$ \\
        \hline
        3 & 0.3 & 0.6 & 0.18 & $145.8*10^-6$ & $30*10^6$ & -10 & -5 \\
        \hline
    \end{tabular}
    \caption{Parametry belki}
\end{table}

\begin{table}[h]
    \centering
    \begin{tabular}{|c|c|c|c|c|c|}
        \hline
        $a$ & $b$ & $A$ & $E$ & $P_x$ & $P_y$ \\
        \hline
        \multicolumn{2}{|c|}{$m$} & $m^2$ & $X$ & \multicolumn{2}{|c|}{$kN$} \\
        \hline
        3 & 3 & 0.2 & $30*10^6$ & 5 & -10 \\
        \hline
    \end{tabular}
    \caption{Paramtery kratownicy}
\end{table}

\subsubsection{Tworzenie węzłów, podpór i elementów}
\subsubsection{Definicja i przyłożenie obciążeń}
\subsubsection{Parametryzacja modelu}
\subsubsection{Przeprowadzenie obliczeń}
\subsubsection{Prezentacja wyników}

\subsection{Optymalizacja kodu obliczeniowego}

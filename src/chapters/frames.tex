\newpage
\section{Modelowanie obliczeniowe konstrukcji}

\subsection{Modelowanie układów ramowych}

\subsection{Implementacja obliczeń}
\subsubsection{Konfiguracja środowiska}

Aby móc zacząć pracę z silnikiem obliczeniowym niezbędne jest prawidłowe skonfigurowanie środowiska programistycznego.
%TODO - add hiperlinks
Ze względu na to, iż komputery nie mają wbudowanego języka Python, ów należy pobrać z oficjalnej strony[1].
Postanowiono pobrać Python wersji 3.11, ponieważ biblioteka \textbf{OpenSeesPy} w wersji 3.5.1.12 akurat tego wymaga,
oraz darmowy edytor kodu źródłowego – \textbf{Visual Studio Code}[2], dla wygody napisana kodu.

Po udanej instalacji Pythona stworzono katalog projektu.
Następnie, korzystając z wiersza poleceń, wybrano wcześniej stworzony katalog i wykonano następne polecenia:

\begin{lstlisting}
python3 -m venv venv
source venv/bin/activate
\end{lstlisting}

W wyniku w wybranym katalogu powstało wirtualne środowisko o nazwie \textbf{venv} w celu odizolować zależności projektu.
Następnie dodano pliki o rozszerzeniu \textbf{.py} dla każdej konstrukcji oraz jeden plik pomocniczy, zawierający wspólny
dla każdego procesu obliczeniowego kod.
Na tym etapie kończy się konfiguracja środowiska.

\subsubsection{Instalacja i import modułów}

Korzystając z wiersza poleceń i komendy \inlinecode{pip install <module name>} zainstalowano następujące biblioteki:

\begin{itemize}
    \item OpenSeesPy
    \item OpsVis
    \item SciPy
    \item NumPy
    \item MatPlotLib
\end{itemize}

Po udanej instalacji przy pomocy komendy \textbf{import} zaimportowano moduły do każdego pliku:

\begin{lstlisting}
# Plik shared.py
import numpy as np
import matplotlib.pyplot as plt
\end{lstlisting}

\begin{lstlisting}
# Pliki obliczeniowe
import numpy as np
import opsvis as opsv
from scipy.stats import norm
import openseespy.opensees as ops
from shared import show_results
\end{lstlisting}

\subsubsection{Tworzenie modelu obliczeniowego i definiowanie parametrów}
\subsubsection{Tworzenie węzłów, podpór i elementów}
\subsubsection{Definicja i przyłożenie obciążeń}
\subsubsection{Parametryzacja modelu}
\subsubsection{Przeprowadzenie obliczeń}
\subsubsection{Prezentacja wyników}

\subsection{Optymalizacja kodu obliczeniowego}

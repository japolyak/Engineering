\section{Przykłady}

W celu zbadania wpływu niepewności parametrów na zachowanie konstrukcji, przeprowadzono serię symulacji Monte Carlo.
W ramach analizy, losowe wartości wybranych parametrów zostały wygenerowane w przedziale \pm 10\% od ich wartości nominalnych.

Aby dokładnie przeanalizować proces stabilizacji wyników oraz wpływ liczby prób na dokładność analizy,
dla każdej konstrukcji przeprowadzono trzy etapy symulacji, obejmujące kolejno 10.000, 100.000 oraz 1.000.000 prób.

Głównym celem tych symulacji jest nie tylko określenie, jak zmiany parametrów wpływają na przemieszczenie konstrukcji,
ale także zbadanie, jak wyniki symulacji zbiegają się do stabilnych wartości wraz ze wzrostem liczby prób w ramach metody Monte Carlo.
To pozwoli na ocenę, ile prób jest niezbędnych do uzyskania wiarygodnych i stabilnych wyników analizy wrażliwości.

\newpage
\subsection{Belka}
Rozpatrywana jest belka swobodnie podparta, składająca się z trzech elementów, wykonanych z jednorodnego materiału o jednakowych właściwościach fizycznych i geometrycznych:
moduł sprężystości \textbf{E}, moment bezwładności przekroju \textbf{I} oraz powierzchnia przekroju poprzecznego \textbf{A}.
Belka jest podparta na dwóch podporach przegubowo nieprzesuwnych oraz jednej podporze przegubowo przesuwnej.

Konstrukcję poddano działaniu dwóch rodzajów obciążeń:

\begin{itemize}
    \item Obciążenia równomiernie rozłożonego q na całej długości pierwszego elementu, skierowanego pionowo w dół.
    \item Obciążeniea punktowego P przyłożonego na końcu belki (węzeł 4), skierowanego pionowo w dół.
\end{itemize}

Na Rysunku \ref{cadmodel:Belka-mod} jest przedstawiony model belki swobodnie podpartej, a Tabela \ref{tab:parametry-belka} zawiera deterministyczne oraz losowane parametry.

Analiza skupia się na przemieszczeniu węzła 4 w kierunku pionowym (w osi Y).

\cadmodel
{Belka-mod}
{Model belki swobodnie podpartej}
{0.35}

\begin{table}[H]
    \centering
    \begin{tabular}{|c|c|c|c|c|c|c|c|c|}
        \hline
        Rodzaj parametru & L & b & d & A & $\mathrm{I}_\mathrm{z}$ & E & P & q \\
        \hline
        – & \multicolumn{3}{|c|}{[m]} & [$\mathrm{m}^\mathrm{2}$] & [$\mathrm{m}^\mathrm{4}$] & [GPa] & [kN] & [kN/m] \\
        \hline
        Deterministyczny & 3 & 0.3 & 0.6 & 0.18 & – & – & – & – \\
        \hline
        Średni & – & – & – & – & 145.8 \cdot $\mathrm{10}^{\mathrm{-6}}$ & 30 & –10 & –5 \\
        \hline
        Odchylenie standardowe & – & – & – & – & 14.6 \cdot $\mathrm{10}^{\mathrm{-6}}$ & 3 & 1 & 0.5 \\
        \hline
    \end{tabular}
    \caption{Parametry belki swobodnie podpartej}
    \label{tab:parametry-belka}
\end{table}

\newpage
\constructionresults
{\belka}
{Belka}
{Ugięcie na końcu belki}
{
\resultstable
{-0.66 & -40.43 & -0.037 & 0.67 & 0.45 & -18.46}
{-0.48 & -556 & -0.045 & 2.15 & 4.61 & -47.6}
{-0.02 & -832.8 & -0.038 & 1.85 & 3.43 & -48.88}
{Wyniki symulacji dla belki}
{Belka-results}
{[μm] & [mm] & [mm] & [mm] & [$\mathrm{mm}^\mathrm{2}$] & [\%]}
}
{
Rysunek \ref{fig:Belka_model} przedstawia model obliczeniowy belki swobodnie podpartej wygenerowany w środowisku symulacyjnym.
Na Rysunku \ref{fig:Belka_deformation} ukazano deformację belki pod wpływem siły skupionej pionowej.
Tabela \ref{tab:Kratownica-results} zawiera wartości przemieszczeń uzyskane w 3 seriach obliczeń.

Warto zauważyć, że wraz ze wzrostem liczby prób symulacyjnych, różnica między maksymalnym a minimalnym przemieszczeniem wzrasta, mimo że dla każdej serii pomiarów jest ona zbyt wielka.
}
{
Rysunki \ref{fig:Belka_10000_sp}, \ref{fig:Belka_10000_zsp} oraz \ref{fig:Belka_10000_zos} prezentują wyniki przemieszczenia węzła 4 dla 10.000 symulacji.
Średnie przemieszczenie wynosi 0.037 mm, a różnica między maksymalną wartością (-0.66 μm) a minimalną (-40.43 mm) wynosi prawie tyle samo co wartość minimalna.
Na Rysunku \ref{fig:Belka_10000_sp} widoczne są przemieszczenia znacząco odbiegające od średniej, co może świadczyć o występowaniu zaburzeń.
Z kolei na Rysunkach \ref{fig:Belka_10000_zsp} – \ref{fig:Belka_10000_zos} widać, że do stabilizacji wyników jeszcze nie doszło.
}
{
Wyniki dla 100.000 symulacji przedstawiono na Rysunkach \ref{fig:Belka_100000_sp}, \ref{fig:Belka_100000_zsp} oraz \ref{fig:Belka_100000_zos}.
Średnie przemieszczenie zwiększyło się do 0.045 mm, a różnica między wartością maksymalną (-0.48 μm) a minimalną (-556 mm) nadal jest zbyt wielka.
Na Rysunku \ref{fig:Belka_100000_sp} nadal obserwowane są przemieszczenia, które mogą być uznane za zaburzenia.
Stabilizacje średniej wartości przemieszczenia oraz odchylenia standardowego (Rysunek \ref{fig:Belka_100000_zsp} – \ref{fig:Belka_100000_zos}) nie są jezcze widoczne.
}
{
Dla 1.000.000 symulacji, wyniki na  Rysunkach \ref{fig:Belka_1000000_sp}, \ref{fig:Belka_1000000_zsp} oraz \ref{fig:Belka_1000000_zos} świadczą, że średnie przemieszczenie wynosi 0.038 mm, a różnica między wartością maksymalną (-0.02 μm) a minimalną (-828.8 mm) jescze bardziej wzrosła.
Na Rysunku \ref{fig:Belka_1000000_sp} wciąż widoczne są wartości przemieszczeń, które mogą być traktowane jako zaburzenia.
Całkowitą stabilizację średniej wartości przemieszczenia można zaobserwować po około 800.000 symulacji (Rysunek \ref{fig:Belka_1000000_zsp}),
natomiast stabilizacja odchylenia standardowego niby następuje dopiero po około 1.00.000 prób (Rysunek \ref{fig:Belka_1000000_zos}).
}


\newpage
\subsection{Rama}
Rozważana jest rama przestrzenna składająca się z trzech elementów belkowych.
Wszystkie elementy wykonane są z jednorodnego materiału, o jednakowych właściwości fizyczne oraz geometryczne(moduł sprężystości
\textbf{E}, moment bezwładności przekroju \textbf{I},powierzchnia przekroju poprzecznego \textbf{A}).

Rama jest podparta z lewej strony przegubowo nieprzesuwnie, a z prawej ma wspornik.
Na lewy górny narożnik (węzeł 2) działają dwie siły skupione: $P_1$ (pozioma, skierowana w prawo) i $P_2$ (pionowa, skierowana w dół).

Głównym celem analizy jest ocena wpływu tych obciążeń na przemieszczenie narożnika w kierunku poziomym (w osi X).

\cadmodel
{Rama-mod}
{Model ramy przestrzennej}

\begin{table}[H]
    \centering
    \begin{tabular}{|c|c|c|c|c|c|c|c|c|}
        \hline
        $L$ & $H$ & $b$ & $d$ & $A$ & $I_z$ & $E$ & $P_1$ & $P_2$ \\
        \hline
        \multicolumn{4}{|c|}{$m$} & $m^2$ & $m^4$ & $GPa$ & \multicolumn{2}{|c|}{$kN$} \\
        \hline
        3 & 4 & 0.3 & 0.6 & 0.18 & $145.8*10^-6$ & $30*10^6$ & 25 & -500 \\
        \hline
    \end{tabular}
    \caption{Parametry ramy}
    \label{tab:pars-rama}
\end{table}

\constructionresults
{\rama}
{Rama}
{Ugięcie środka belki}
{
\resultstable
{1.33 & 0.27 & 1.07 & 0.17 & 2.98 & 0.16}
{1.36 & 0.26 & 1.07 & 0.17 & 3 & 0.16}
{1.39 & 0.24 & 1.07 & 0.17 & 3 & 0.16}
{Wyniki przemieszczeń symulacji dla ramy}
{rama-results}
{$[m] 10^{-3}$ & $[m] 10^{-3}$ & $[m] 10^{-3}$ & $[m] 10^{-3}$ & $[m] 10^{-8}$ & $\%$}
}


\newpage
\subsection{Kratownica}
Przedmiotem analizy jest kratownicowa składająca się z trzech elementów prętowych, przy czym każdy z prętów jest wykonany
z innego materiału, co oznacza, że mają one różne moduły sprężystości \textbf{E} oraz powierzchnie przekroju poprzecznego \textbf{A}.

Każdy z elementów kratownicy jest zamocowany u góry za pomocą przegubu nieprzesuwnego, a w węźle 4 elementy są połączone między sobą przegubowo.

Konstrukcja poddana jest działaniu siły skupionej \textbf{P} przyłożonej w węźle 4, skierowanej skośnie w prawy dół.
W implementacji siłę \textbf{P} rozłożono na składowe w płaszczyźnie Y oraz X.

Na Rysunku \ref{cadmodel:Kratownica-mod} jest przedstawiony model kratownicy, a Tabela \ref{tab:parametry-kratownica} zawiera deterministyczne oraz losowane parametry konstrukcji.

Analiza skupia się na ocenie przemieszczenia węzła 4 w kierunku pionowym (w osi Y) pod wpływem działania siły \textbf{P}.

\cadmodel
{Kratownica-mod}
{Model kratownicy}
{0.35}

\begin{table}[H]
    \centering
    \begin{tabular}{|c|c|c|c|c|c|c|c|c|c|c|c|}
        \hline
        Rodzaj parametru & a & b & c & $\mathrm{E}_\mathrm{1}$ & $\mathrm{E}_\mathrm{2}$ & $\mathrm{E}_\mathrm{3}$ & $\mathrm{A}_\mathrm{1}$ & $\mathrm{A}_\mathrm{2}$ & $\mathrm{A}_\mathrm{3}$ & $\mathrm{P}_\mathrm{x}$ & $\mathrm{P}_\mathrm{y}$ \\
        \hline
        – & \multicolumn{3}{|c|}{[m]} & \multicolumn{3}{|c|}{[GPa]} & \multicolumn{3}{|c|}{[$\mathrm{m}^\mathrm{2}$]} & \multicolumn{2}{|c|}{[kN]} \\
        \hline
        Deterministyczny & 3 & 4 & 5 & 30 & 40 & 50 & – & – & – & – & – \\
        \hline
        Średni & – & – & – & – & – & – & 0.3 & 0.2 & 0.1 & 15 & –5 \\
        \hline
        Odchylenie standardowe & – & – & – & – & – & – & 0.03 & 0.02 & 0.01 & 1.5 & 0.5 \\
        \hline
    \end{tabular}
    \caption{Parametry kratownicy}
    \label{tab:parametry-kratownica}
\end{table}

\constructionresults
{\kratownica}
{Kratownica}
{Przesunięcie węzła 4}
{
\resultstable
{8.25 & -0.88 & 0.038 & 0.19 & 35.9 & 4.97}
{19.5 & -1.88 & 0.035 & 0.17 & 30.9 & 5.02}
{16.1 & -6 & 0.035 & 0.16 & 24.4 & 4.46}
{Wyniki symulacji dla kratownicy}
{Kratownica-results}
{[mm] & [mm] & [mm] & [mm] & [nm] & [\%]}
}
{
Rysunek \ref{fig:Kratownica_model} przedstawia model obliczeniowy konstrukcji kratownicowej wygenerowany w środowisku symulacyjnym.
Na Rysunku \ref{fig:Kratownica_deformation} ukazano deformację kratownicy pod wpływem siły skośnej, przyłożonej w węźle 4.
Tabela \ref{tab:Kratownica-results} zawiera wartości przemieszczeń uzyskane w 3 seriach obliczeń.

Warto zauważyć, że wraz ze wzrostem liczby prób symulacyjnych, różnica między maksymalnym a minimalnym przemieszczeniem wzrasta.
Ponadto, stosunek minimalnej wartości przemieszczenia, najbardziej zbliżonej do wartości średniej, do samej wartości średniej rośnie z 20 do 170 w miarę zwiększania liczby symulacji.
}
{
Rysunki \ref{fig:Kratownica_10000_sp}, \ref{fig:Kratownica_10000_zsp} oraz \ref{fig:Kratownica_10000_zos} prezentują wyniki przemieszczenia węzła 4 dla 10.000 symulacji.
Średnie przemieszczenie wynosi 0.038 mm, a różnica między maksymalną wartością (8.25 mm) a minimalną (-0.88 mm) wynosi 9.13 mm.
Na Rysunku \ref{fig:Kratownica_10000_sp} widoczne są przemieszczeń znacząco odbiegające od średniej, co może świadczyć o występowaniu zaburzeń.
Z kolei na Rysunkach \ref{fig:Kratownica_10000_zsp} – \ref{fig:Kratownica_10000_zos} widać, że stabilizacja wyników jeszcze nie nastąpiła.
}
{
Wyniki dla 100,000 symulacji przedstawiono na Rysunkach \ref{fig:Kratownica_100000_sp}, \ref{fig:Kratownica_100000_zsp} oraz \ref{fig:Kratownica_100000_zos}.
Średnie przemieszczenie wynosi 0.035 mm, a różnica między wartością maksymalną (19.5 mm) a minimalną (-1.88 mm) wzrosła do 21.38 mm.
Na Rysunku \ref{fig:Kratownica_100000_sp} nadal obserwowane są pojedyncze przemieszczenia, które mogą być uznane za zaburzenia.
Stabilizacja średniej wartości przemieszczenia jest zauważalna po około 60,000 próbach (Rysunek \ref{fig:Kratownica_100000_zsp}),
 jednak stabilizacja odchylenia standardowego (Rysunek \ref{fig:Kratownica_100000_zos}) nie jest jezcze widoczna.
}
{
Dla 1,000,000 symulacji, wyniki na  Rysunkach \ref{fig:Kratownica_1000000_sp}, \ref{fig:Kratownica_1000000_zsp} oraz \ref{fig:Kratownica_1000000_zos} pokazują, że średnie przemieszczenie wynosi 0.035 mm, a różnica między wartością maksymalną (16.1 mm) a minimalną (-6 mm) to 22.1 mm.
Na Rysunku \ref{fig:Kratownica_1000000_sp} wciąż widoczne są wartości przemieszczeń, które mogą być traktowane jako zaburzenia.
Całkowitą stabilizację średniej wartości przemieszczenia można zaobserwować po około 100,000 symulacji (Rysunek \ref{fig:Kratownica_1000000_zsp}),
natomiast stabilizacja odchylenia standardowego następuje dopiero po około 900,000 prób, co jest widoczne na Rysunku \ref{fig:Kratownica_1000000_zos}.
}
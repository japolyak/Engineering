\section{Przykłady}

W celu zbadania wpływu niepewności parametrów na zachowanie konstrukcji, przeprowadzono serię symulacji Monte Carlo.
W ramach analizy, losowe wartości wybranych parametrów zostały wygenerowane w przedziale \pm 10\% od ich wartości nominalnych.

Aby dokładnie przeanalizować proces stabilizacji wyników oraz wpływ liczby prób na dokładność analizy,
dla każdej konstrukcji przeprowadzono trzy etapy symulacji, obejmujące kolejno 10.000, 100.000 oraz 1.000.000 prób.

Głównym celem tych symulacji jest nie tylko określenie, jak zmiany parametrów wpływają na przemieszczenie konstrukcji,
ale także zbadanie, jak wyniki symulacji zbiegają się do stabilnych wartości wraz ze wzrostem liczby prób w ramach metody Monte Carlo.
To pozwoli na ocenę, ile prób jest niezbędnych do uzyskania wiarygodnych i stabilnych wyników analizy wrażliwości.

\newpage
\subsection{Rama}
\begin{table}[H]
    \centering
    \begin{tabular}{|c|c|c|c|c|c|c|c|c|}
        \hline
        $L$ & $H$ & $b$ & $d$ & $A$ & $I_z$ & $E$ & $P_1$ & $P_2$ \\
        \hline
        \multicolumn{4}{|c|}{$m$} & $m^2$ & $m^4$ & $X$ & \multicolumn{2}{|c|}{$kN$} \\
        \hline
        3 & 4 & 0.3 & 0.6 & 0.18 & $145.8*10^-6$ & $30*10^6$ & 150 & -50 \\
        \hline
    \end{tabular}
    \caption{Parametry ramy}
    \label{tab:pars-rama}
\end{table}

\resultstable
{1 & 2 & 3 & 4 & 5 & 6}
{1 & 2 & 3 & 4 & 5 & 6}
{1 & 2 & 3 & 4 & 5 & 6}
{Wyniki symulacji dla ramy}
{rama-results}

\opsfigure{Rama_model}{Model konstrukji – Rama}
\opsfigure{Rama_10000_def}{Deformacja konstrukji – Rama 10000}
\opsfigure{Rama_10000_sp}{Ugięcie środka belki – Rama 10000}
\opsfigure{Rama_10000_zsp}{Zbieżność wartości średniej przemieszczenia - Rama 10000}
\opsfigure{Rama_10000_zos}{Zbieżność odchylenia standardowego – Rama 10000}

\opsfigure{Rama_100000_def}{Deformacja konstrukji – Rama 100000}
\opsfigure{Rama_100000_sp}{Ugięcie środka belki – Rama 100000}
\opsfigure{Rama_100000_zsp}{Zbieżność wartości średniej przemieszczenia - Rama 100000}
\opsfigure{Rama_100000_zos}{Zbieżność odchylenia standardowego – Rama 100000}

\opsfigure{Rama_1000000_def}{Deformacja konstrukji – Rama 1000000}
\opsfigure{Rama_1000000_sp}{Ugięcie środka belki – Rama 1000000}
\opsfigure{Rama_1000000_zsp}{Zbieżność wartości średniej przemieszczenia - Rama 1000000}
\opsfigure{Rama_1000000_zos}{Zbieżność odchylenia standardowego – Rama 1000000}


\newpage
\subsection{Kratownica}
Przedmiotem analizy jest kratownicowa składająca się z trzech elementów prętowych, przy czym każdy z prętów jest wykonany
z innego materiału, co oznacza, że mają one różne moduły sprężystości \textbf{E} oraz powierzchnie przekroju poprzecznego \textbf{A}.

Każdy z elementów kratownicy jest zamocowany u góry za pomocą przegubu nieprzesuwnego, a w węźle 4 elementy są połączone między sobą przegubowo.

Konstrukcja poddana jest działaniu siły skupionej \textbf{P} przyłożonej w węźle 4, skierowanej skośnie w prawy dół.
W implementacji siłę \textbf{P} rozłożono na składowe w płaszczyźnie Y oraz X.

Analiza skupia się na ocenie przemieszczenia węzła 4 w kierunku pionowym (w osi Y) pod wpływem działania siły \textbf{P}.

\cadmodel
{Kratownica-mod}
{Model kratownicy}

\begin{table}[H]
    \centering
    \begin{tabular}{|c|c|c|c|c|c|c|c|c|c|c|c|}
        \hline
        Rodzaj parametru & a & b & c & $\mathrm{E}_\mathrm{1}$ & $\mathrm{E}_\mathrm{2}$ & $\mathrm{E}_\mathrm{3}$ & $\mathrm{A}_\mathrm{1}$ & $\mathrm{A}_\mathrm{2}$ & $\mathrm{A}_\mathrm{3}$ & $\mathrm{P}_\mathrm{x}$ & $\mathrm{P}_\mathrm{y}$ \\
        \hline
        – & \multicolumn{3}{|c|}{[m]} & \multicolumn{3}{|c|}{[GPa]} & \multicolumn{3}{|c|}{[$\mathrm{m}^\mathrm{2}$]} & \multicolumn{2}{|c|}{[kN]} \\
        \hline
        Deterministyczny & 3 & 4 & 5 & 30 & 40 & 50 & – & – & – & – & – \\
        \hline
        Średni & – & – & – & – & – & – & 0.3 & 0.2 & 0.1 & 15 & –5 \\
        \hline
        Odchylenie standardowe & – & – & – & – & – & – & 0.03 & 0.02 & 0.01 & 1.5 & 0.5 \\
        \hline
    \end{tabular}
    \caption{Parametry kratownicy}
    \label{tab:pars-kp}
\end{table}

\constructionresults
{\kratownica}
{Kratownica}
{Przesunięcie węzła 4}
{
\resultstable
{8.25 & -0.88 & 38.1 & 0.19 & 35.9 & 4.97}
{19.5 & -1.88 & 35 & 0.17 & 30.9 & 5.02}
{16.1 & -6 & 35 & 0.16 & 24.4 & 4.46}
{Wyniki symulacji dla kratownicy}
{kp-results}
{[mm] & [mm] & [μm] & [mm] & [nm] & [\%]}
}


\newpage
\subsection{Belka}
\begin{table}[H]
    \centering
    \begin{tabular}{|c|c|c|c|c|c|c|c|}
        \hline
        $L$ & $b$ & $d$ & $A$ & $I_z$ & $E$ & $P$ & $q$ \\
        \hline
        \multicolumn{3}{|c|}{$m$} & $m^2$ & $m^4$ & $X$ & $kN$ & $kN/m$ \\
        \hline
        3 & 0.3 & 0.6 & 0.18 & $145.8*10^-6$ & $30*10^6$ & -10 & -5 \\
        \hline
    \end{tabular}
    \caption{Parametry belki}
    \label{tab:pars-belka}
\end{table}

\resultstable
{1 & 2 & 3 & 4 & 5 & 6}
{1 & 2 & 3 & 4 & 5 & 6}
{1 & 2 & 3 & 4 & 5 & 6}
{Wyniki symulacji dla belki}
{belka-results}

\opsfigure{Belka_model}{Model konstrukji – Belka}
\opsfigure{Belka_10000_def}{Deformacja konstrukji – Belka 10000}
\opsfigure{Belka_10000_sp}{Ugięcie na końcu belki – Belka 10000}
\opsfigure{Belka_10000_zsp}{Zbieżność wartości średniej przemieszczenia - Belka 10000}
\opsfigure{Belka_10000_zos}{Zbieżność odchylenia standardowego – Belka 10000}

\opsfigure{Belka_100000_def}{Deformacja konstrukji – Belka 100000}
\opsfigure{Belka_100000_sp}{Ugięcie na końcu belki – Belka 100000}
\opsfigure{Belka_100000_zsp}{Zbieżność wartości średniej przemieszczenia - Belka 100000}
\opsfigure{Belka_100000_zos}{Zbieżność odchylenia standardowego – Belka 100000}

\opsfigure{Belka_1000000_def}{Deformacja konstrukji – Belka 1000000}
\opsfigure{Belka_1000000_sp}{Ugięcie na końcu belki – Belka 1000000}
\opsfigure{Belka_1000000_zsp}{Zbieżność wartości średniej przemieszczenia - Belka 1000000}
\opsfigure{Belka_1000000_zos}{Zbieżność odchylenia standardowego – Belka 1000000}

